\documentclass[parskip=full]{scrartcl}
%\documentclass[journal,article,submit,moreauthors,pdftex]{mdpi}

\pdfoutput=1

\title{Imbalanced Learning in Land Cover Classification  \\ \LARGE{Improving minority classes' prediction accuracy using the Geometric SMOTE algorithm}}

\author{
	Georgios Douzas\(^{1}\), Fernando Bacao\(^{1*}\), Joao Fonseca\(^{1}\), Manvel Khudinyan\(^{1}\)
	\\
	\small{\(^{1}\)NOVA Information Management School, Universidade Nova de Lisboa}
	\\
	\small{*Corresponding Author}
	\\
	\\
	\small{Postal Address: NOVA Information Management School, Campus de Campolide, 1070-312 Lisboa, Portugal}
	\\
	\small{Telephone: +351 21 382 8610}
}

\usepackage{colortbl}
\usepackage{breakcites}
\usepackage{float}
\usepackage{graphicx}
\usepackage[colorinlistoftodos]{todonotes}
\usepackage{geometry}
\geometry{
	a4paper,
	total={170mm,257mm},
	left=18mm,
	right=18mm,
	top=8mm,
}
\usepackage{amsmath}
\newcommand{\inlineeqnum}{\refstepcounter{equation}~~\mbox{(\theequation)}}
\usepackage{enumitem}
\usepackage[ruled,vlined]{algorithm2e}
\usepackage{booktabs}
\usepackage{pgfplotstable}
\pgfplotsset{compat=1.14}
\usepackage{longtable}
\usepackage{tabu}
\usepackage{hyperref}
\usepackage{csvsimple}
\date{}

\begin{document}

\maketitle

\begin{abstract}
The classification of Land Use/Land Cover (LULC) maps is one of the main
research topics in current times. Despite its popularity, the classification of
LULC maps has proven to be a difficult task. One of the most common challenges
resides in the data imbalance problem. In this work, we propose Geometric SMOTE
(G-SMOTE) as a means of addressing such challenge. The performance of this
oversampler is tested using a number of classifiers (Logistic Regression, K-
Nearest Neighbors, Decision Trees, Gradient Boosting Classifier and Random
Forests) and oversamplers (SMOTE, Borderline SMOTE, Random Oversampling and
ADASYN). As a baseline, standard classification is applied using no
oversampling. The dataset used consisted of 8 classes with an imbalance ratio
of 190.25, extracted from the LUCAS dataset. The experiment comprised a grid
search over all possible combinations of oversamplers, classifiers and
respective hyperparameters. The performance of these oversamplers are compared
using 3 different metrics, as to provide a holistic approach to the imbalanced
learning problem. When compared to its popular counterpart (SMOTE), G-SMOTE
provides a consistent G-Mean, F- score and overall accuracy increase across all
classifiers. Specifically, the Decision Tree classifier increased its
performance by 6.01\%, 6.98\% and 14.23\%, respectively. Overall, our study
suggests that G-SMOTE performs consistently better than any other oversampling
method.
\end{abstract}

\textbf{Keywords:} Imbalance learning, LULC classification, oversampling,
Geometric-SMOTE, class
imbalance
\end{keywords}

\section{Introduction}

The production of accurate Land Use/Land Cover maps offer unique monitoring
capabilities within the remote-sensing domain \cite{Mellor2015}. LULC maps are
being used for a variety of applications, ranging from environmental
monitoring, land change detection, natural hazard assessment up to agriculture
and water/wetland monitoring \cite{Khatami2016}.

Multispectral images are an important resource to build LULC maps, allowing for
the use of classification algorithms, both supervised and unsupervised, to
automate the production of these maps. Although significant progress has been
made in the use of supervised learning techniques for automatic image
classification \cite{Tewkesbury2015}, the acquisition of labeled training sets
continues to be a bottleneck \cite{Rajan2008}. To build  good and robust
supervised classifiers it is crucial to have a large enough training dataset.
Often, the problem is that different land use classes have very different
levels of coverage, while some are frequent in the training dataset, other can
be limited \cite{Feng2019}. In remote-sensing classification problems, the
number of instances of some classes is often very small when compared with the
dominant classes, this is especially true in LULC data \cite{Williams2009,
Cenggoro2018}.

This imbalance in the representation of the different classes constitutes a
problem for the development of accurate classifiers. The problem is usually
referred to, in the machine learning community, as the imbalanced learning
problem \cite{Chawla2004}. The imbalanced learning problem is even more
challenging in multi-class problems, making it one of the most
important challenges in the machine learning research community
\cite{Garcia2018}. Improvements in the representation of small classes in
training datasets can have a significant impact in improving the accuracy and
robustness of classifiers for the production of LULC maps.

An imbalanced learning problem refers to a skewed distribution of data across
classes in both binary and multi-class classification problems \cite{Abdi2016}.
In this situation, the minority classes contribute less to the minimization of
accuracy, the typical objective function, inducing a bias towards the majority
class during the learning process \cite{Douzas2019}. Consequently, as typical
classifier learning methods are designed to work with reasonably balanced
datasets, finding meaningful boundaries between the different classes becomes a
very difficult task \cite{Saez2016}.

Possible approaches to deal with class imbalance can be divided into three main
groups \cite{Fernandez2013}. 1) Cost-sensitive solutions: adaptations at
algorithmic and/or data level by applying higher misclassification costs for the
examples of the minority class. 2) Algorithmic level solutions: classification
algorithms are adapted or created to reinforce the learning of the positive
class. 3) Data level solutions: a more general approach where class
distributions are rebalanced through the sampling of the data space, thus
diminishing class imbalance. Because this latter method is more general, it is
used for a wide range of applications, thus making it of particular interest.
Regardless, data level solutions carry the disadvantage of yielding an increased
search space.

There are several data level solutions to deal with the imbalanced learning
problem, which can be divided into three groups. While Undersampling algorithms
reduce the size of the majority classes, oversampling algorithms attempt to
even the distributions by generating artificial data for the minority class
\cite{Mellor2015}. A hybrid approach, on the other hand, uses both oversampling
and undersampling techniques to ensure a balanced dataset.

Studies suggest that the usage of data level solutions in remote sensing
problems, to balance class distribution, seem to outperform models
trained by randomly drawn samples \cite{Wang2019, Mellor2015}.  Studies
employing oversampling methods such as the Synthetic Minority Over-sampling
Technique (SMOTE) \cite{Chawla2002} seem to consistently yield better results
\cite{Johnson2013, Geib2015} when compared to no oversampling. Other studies
employ active learning methods such as Margin Sample by Closest Support Vector
(MS-cSV), as it benefits from avoiding oversampling in dense regions close to
the margin and samples all the feature space equivalently \cite{Tuia2009}.
In spite of the potential benefits that oversampling techniques can bring to
the improvement of automatic classification, there are not many studies on the
subject in the remote sensing domain.

In this paper, we test the performance of oversampling techniques on an
imbalanced dataset using different types of classifiers. The effectiveness of
the Geometric SMOTE (G-SMOTE) \cite{Douzas2019} is demonstrated using SMOTE
\cite{Chawla2002}, Borderline SMOTE \cite{Han2005}, ADASYN \cite{HaiboHe2008},
Random Oversampling and no oversampling as baseline methods. Results
show that G-SMOTE outperforms every other oversampling technique, for both
overall prediction power and minority class prediction power.

Oversamplers' performance is evaluated through an experimental analysis using
the publicly available EUROSTAT´s Land Use/Cover Area Statistical Survey
(LUCAS) dataset \cite{LUCAS2015}. This dataset contains multispectral
reflectance data from Landsat 8, filtered for a region in Portugal. The 
experimental procedure includes a comparison of the different oversamplers
using 5 classifiers and 3 evaluation metrics.

This paper is organized in 5 sections: Section 2 analyses the state of the art,
section 3 describes the proposed methodology, section 4 shows the results and
discussion and section 5 presents the conclusions drawn from this study.

\section{State of the Art}

Data modification through resampling has been the most popular approach to deal
with the imbalanced learning problem, in machine learning in general
\cite{Douzas2019} and remote sensing in particular \cite{Feng2019}. By
decoupling the imbalance problem from the classification algorithms, resampling
allows the users to apply any standard algorithm once the resampling
preprocessing step is done. This is especially convenient for users that are
not machine learning experts and want to use several classifiers. Additionally,
resampling methods can be easily adapted to multi-class imbalanced data, which
is relevant for LULC classification \cite{Feng2019}. In this section we present
the most relevant applications of resampling methods for remote sensing
imbalanced data classification.

Some of the existing studies implements the random undersampling method,
which randomly reduces the number of training samples  belonging to the
majority class. However, this method has the disadvantage of information loss
as it discards samples from the majority class  \cite{Feng2019}.
\cite{Feng2018} carries out competitive analysis of  undersamplers and
oversamplers performance for land cover classification with  Rotation Forest
(RoF) ensemble classifier. In the experiment random  undersampling,
undersampling based RoF and a number of oversampling techniques  were deployed.
The used dataset comprises 16 imbalanced classes from  hypercritical imagery.
The results reported show the limited interest of the  undersampling methods as
they always produce worst results in the prediction of all the 16 classes.

Contrary to random undersampling, the random oversampling method avoids
information loss. However, this method simply replicates randomly selected
instances of the minority class, consequently, increasing the risk of
over-fitting \cite{Krawczyk2016}. \cite{Maxwell2018} reports that balancing
data with random oversampling affects the classification performance
differently for different supervised models. In their paper, land cover
classification with highly imbalanced data is carried out with six different
supervised classifiers. Implementation of random oversampling could slightly
improve the performance of Random Forest (RF) and Support Vector Machine (SVM)
classifiers. On the other hand, it reduced the overall classification accuracy
for classifiers such as Decision Tree (DT), Artificial Neural Network (ANN),
K-Nearest Neighbors (KNN) and Boosted DT. The paper shows, that even if random
oversampling can positively impact the prediction of minority classes,
it can also reduce the accuracy in the majority class.

Synthetic minority oversampling technique (SMOTE) \cite{Chawla2002} is the most
popular oversampling method, and it has been used to successfully deal with the
class imbalance problem in land cover classification. The variational
semi-supervised learning (VSSL) proposed by \cite{Cenggoro2018} aims to fix the
imbalance problem in LULC mapping. VSSL is a semi-supervised learning framework
consisting of a deep generative model allowing learning from both labeled
and unlabeled samples while using SMOTE to balance the data. The results show
significant improvements in performance on the producer’s accuracy of the
minority class, when compared to the results without any imbalanced learning.

Ensemble learning method together with oversampling algorithms have been
successfully implemented for remote sensing image classification in the resent
years. \cite{Feng2018, Feng2019} apply the Rotation Forest ensemble
learning algorithm with SMOTE to deal with class imbalance problem. Those
papers offer a novel approach to the use of SMOTE based RoF algorithm for a
hyperspectral image classification. The idea is to iteratively balance the
class distribution using SMOTE before creating each rotation decision tree.
This approach yielded improved classification accuracy in the
minority classes,  as well as for the whole dataset.

A number of studies report significant improvements in LULC mapping accuracy
with the use of SMOTE oversampling along with standard algorithms to handle
data imbalance for multi-source remote sensing data. \cite{Johnson2016} uses
OpenStreetMap crowdsourced data and Landsat time series for LULC
classification. A simple implementation of the SMOTE oversampler shows improved
classification results for all the supervised classifiers (Naive Bayes, DT and
RF). \cite{Bogner2018} used a RF classifier to detect rare land use classes on
MODIS time series. They use SMOTE to decrease the imbalance ratio in the
original dataset. The comparison of different classification scenarios shows
the minority classes to be better classified after SMOTE. \cite{Panda2018}
carried out a simple implementation of SMOTE with five different classifiers
(Naive Bayes, DT, RF, KNN, SVM) to predict a land cover maps with 4, 5 and 6
classes. The results show the relevance of using oversampling to improve the
classifiers' performance, as 5 of the classifiers improve their results.

Even if recent studies demonstrate the usefulness of oversampling methods
for remote sensing applications, they still present some drawbacks. Random
oversampling often leads to the model over-fitting as it simply replicates the
instances for minority class \cite{Feng2019}. SMOTE method can avoid
over-fitting but has the disadvantage of generating noisy data \cite{He2008}.
In order to mitigate this problem many variations of SMOTE have been developed
with the objective of preventing noisy data generation (see section 1). In this
study we use a set of more recent oversamplers, most notably Geometric-SMOTE,
that is yet to be adopted within the remote sensing community.

The novel oversampling technique Geometric-SMOTE algorithm is an extension of
SMOTE algorithm and has been proposed recently by \cite{Douzas2019}. Similar to
SMOTE, the size of the area for data generation for Geometric-SMOTE is derived
from the distance of the selected minority sample to the nearest neighbor using
$k$-neighbors. The main difference to SMOTE is that instead of using line
segment between minority samples, Geometric-SMOTE defines a flexible geometric
region (hyper-spheroid) around each minority class instance for synthetic data
generation. The shape of geometric regions are defined with $deformation
factor$ and $truncation factor$ hyper-parameters. This intensively increases
the options for more informative data generation. Furthermore, Geometric-SMOTE
is designed to avoid noisy sample generation by some restrictions in the
neighbor selection strategy. This oversampling method has already proven to
outperform SMOTE and random oversampling algorithms across 69 imbalanced
datasets and number of classifiers such as logistic regression, K-Nearest
Neighbors, Decision Trees and Gradient Boosting Classifier.

Borderline-SMOTE is one of the most popular SMOTE based oversamplers. It has two
variations, Borderline-SMOTE1 and Borderline-SMOTE2 and focuses on oversampling
only the minority instances near the borderline to create better boundaries
between classes \cite{Han2005}. As in the case of SMOTE, it calculates $k$
nearest neighbors for all the minority instances. Next, only borderline
minority samples are selected as DANGER set if the majority of the $k$ nearest
neighbors belong to the majority class. Borderline-SMOTE1 method uses all
instances from DANGER set to calculates $s$ nearest neighbors from minority
class and generate synthetic instances with a simple SMOTE method. Borderline-
SMOTE2 uses not only $s$ minority neighbors for each instance of DANGER set,
but also the nearest majority class neighbors. For this, the difference between
the DANGER set instance and majority class instance is multiplied with a random
number between 0 and 0.5 to ensure that the new generated instance is located
closer to the minority class. Borderline-SMOTE has also been reported to
perform better than SMOTE in number of studies \cite{Nguyen2009, Ramentol2012}.

The Adaptive Synthetic Sampling Technique (ADASYN) is another well known
oversampling method based on SMOTE. Adaptive Synthetic sampling is based on the
idea of adaptively generating minority class instances according to their
weighted distribution: more instances are generated for those minority class
instances that are harder to learn compared to ones that are easier to learn
\cite{HaiboHe2008}. With this characteristics ADASYN differs from SMOTE method,
which gives equal chances for all the minority instances to be selected for
synthetic data generation. The synthetic instances are created based on the
majority nearest neighbors using $k$ nearest neighbor method. ADASYN
oversampling method has the advantage of reducing the learning bias introduced
by the original imbalance data distribution. One drawback of this approach is
that it does not identify noisy instances, and thus it does not prevent from
adding outlier values in the dataset.

To compare the performance of the different oversamplers we use the LUCAS
dataset and 5 different supervised classifiers: Logistic Regression, K-Nearest
Neighbours, Decision Trees, Gradient Boosting Classifier and Random Forest.

\section{Methodology}

This section reports the evaluation process of G-SMOTE's performance. A
description of the baseline methods, classification algorithms, evaluation
methods, dataset used and experimental procedure is provided.

The implementation of the experimental procedure was based on the Python
programming language, using the Scikit-Learn \cite{Pedregosa2011}, Imbalanced-
Learn \cite{JMLR:v18:16-365} and Geometric-SMOTE \cite{Douzas2019} libraries.
The experiments reported in this paper as well as the analysis of the results
are reproducible using the scripts available at \url{https://github.com/AlgoWit/
publications}.

\subsection{Baseline methods}

Four oversampling techniques were used in this experiment as baseline methods
along with G-SMOTE. Random oversampling was chosen for its simplicity.
SMOTE was selected for being a classic technique and the most popular
oversampling algorithm \cite{Douzas2019}. ADASYN \cite{HaiboHe2008} and
Borderline SMOTE \cite{Han2005} were selected for representing popular
modifications of the original SMOTE technique in the selection phase and data
generation mechanism, respectively. Finally, we use no oversampling as an
additional baseline method.

\subsection{Classification algorithms}

The selection of classification algorithms was done according to 3 criteria:
learning type, training time and popularity within the remote sensing
community. We further divide classification algorithms into 4 different learning
types: neighbours-based, rule-based, ensemble and generalised linear methods.

For each combination of oversampler and classifier (represented as $x_a$ and
$y_b$, respectively), a grid search will be run in order to find the optimal
parameter settings. The number of fits for a single combination  is defined as:

\begin{equation}
fits_{x_a,y_b}=\prod\limits_{p=1} \textrm{count}(x_a^p).\prod\limits_{q=1}
\textrm{count}(y_b^q)
\end{equation}

Where $\textrm{count}(x_a^p)$ corresponds to the search grid size of
hyperparameter $p$ of an oversampler $x_a$. Finally, the total number of fits
is defined as:

\begin{equation}
Total\ Fits=k\sum\limits_{a=1} \sum\limits_{b=1} fits_{x_a,y_b}
\end{equation}

Where $k$ is the number of folds defined for cross-validation. Thus, adding new
classifiers, oversamplers or grid search values may exponentially increase the
time necessary to run the experiment. Therefore, in order to setup an extensive
grid search and run the full experiment within feasible time, selecting
computationally efficient classifiers becomes vital.

\cite{Khatami2016} developed a meta-analysis of classification processes out
of 266 manuscripts published from 1998 to 2012. Additionally,
\cite{Maxwell2018} also describe mature classification methods used in remote
sensing. Both manuscripts were used to determine the most commonly used
classifiers for LULC classification problems: K-Nearest Neighbours, Support
Vector Machines, Decision Trees, Boosted Decision Trees, Random Forests and
Neural Networks. None of the manuscripts mention the use of generalised linear
models.

Amongst the neighbours-based learning methods the KNN algorithm was selected.
Out of the rule-based learning methods, the decision tree classifier (DT) was
chosen. Two ensemble learning methods were chosen: Gradient boosting classifier
and Random Forest classifier. Finally, the logistic regression classifier was
selected amongst the generalised linear models, since it is a commonly used
baseline algorithm. All these algorithms were found to be computationally
efficient and commonly used classifiers for the proposed task (with exception
for the logistic regression).

Despite their common use, both artificial neural networks and Support Vector
Machines were not included in this experiment due to their long training times.

\subsection{Evaluation methods} \label{evaluation methods}

Amongst the many evaluation metrics existing for classifier's performance
evaluation, there are several indices that are considered to be preliminary and
are widely used for LULC classification accuracy assessment. Those measures are
classification overall accuracy, user's accuracy or precision and
producer's accuracy or recall,  and are calculated using the classification
confusion matrix \cite{Liu2007}. However, the LUCAS dataset has high
inter-class variability. Hence, the minority classes can be difficult to
account for by the classifiers that are designed to maximise the overall
accuracy, as the majority classes are receiving more weight \cite{Inglada2017}.
On the other hand, \cite{He2008} states that the use of user's and producer's
accuracy for class imbalanced data is not straightforward and producer's
accuracy is not sensitive to the data distribution at all. Considering these,
class specific metrics are needed to better describe classifier's performance.
For this purpose, F-score and G-mean metrics are used to evaluate and compare
the performance of the oversampling methods. Classification overall accuracy
(OA) is provided for discussion.

\begin{itemize}

	\renewcommand\labelitemi{--}
	\item Overall Accuracy

	Classification Overall Accuracy is the number of correctly classified
	samples divided by the sum of all the samples on the confusion matrix.

	In case when the training data is imbalanced and the accuracy metric is
	utilized, the result of the grid search for the best parameters will
	indicate that the particular parameterization is the one with highest
	classification overall accuracy, which is biased towards the majority class
	and can disregard the classification of minority classes.

	\item F-score

	F-score (or F-measure) is the harmonic mean of user's accuracy (precision)
	and	producer's accuracy (recall), where user's accuracy is the fraction of
	correctly classified instances with regard to all instances classified as
	this certain class in the confusion matrix, and producer's accuracy is the
	fraction of  correctly classified instances with regard to all instances of
	that reference class. The F-score value for multi-class imbalanced data
	classification can be calculated considering the user's accuracy and
	producer's accuracy as a unweighted average of the respective values from
	all the classes.

	\begin{equation}
	F{-}score=2\frac{\text{E(UA)} \times \text{E(PA)}}{\text{E(UA)} +
	\text{E(PA)}}
	\end{equation}

	Where E(UA) and E(PA) are unweighted mean values of user's and producer's
	accuracy of all the classes, respectively.

	\item G-measure

	The geometric mean score (G-mean) is the root of the product of class-wise
	sensitivity and specificity. For the multi-class case sensitivity and
	specificity are calculated for each class, then averaged with simple mean
	value. Here the sensitivity (true positive rate) is the same with producer's
	accuracy (recall), and specificity (true negative rate) is defined as the
	proportion of actual negatives classified as such. Adequately, specificity
	can be defined as producer's accuracy of the negative class
	\cite{Silva2017}.

	\begin{equation}
	G{-}mean = \sqrt{E(Sensitivity) \times E(Specificity)}
	\end{equation}

	Where E(Sensitivity) and E(Specificity) are the unweighted mean values of
	sensitivity and specificity of all the classes, respectively. The geometric
	mean indicates the balance between classification performances on both
	majority and minority classes. It means, that high misclassification in the
	minority class will lead to a lower geometric mean value. This is a very
	valuable feature for evaluation of the classifier's performance with class
	imbalance problem, which is the case of this study.

\end{itemize}

\subsection{Experiment Settings}

As it is stated in the section 3.2, machine-learning classifiers usually have
parameters that need to be set up by users for a better classification
performance. Parameter tuning to define the optimal values of the
hyper-parameters was performed with 3-fold cross-validation (CV) for each of
the model (combining classifiers with oversampler). The best parameters are
later selected to validate the classifier's performance.

In order to avoid over-optimistic classification results, implementation
of CV with oversampling technique should be carried out as following
\cite{Lusa2015}. In $k$-fold CV dataset is divided into $k$ parts, one part is
excluded and used as only test set, while the rest $k$-1 part is used to train
the model. This process is iterated and each of the $k$ folds are used only
once as a test set to evaluate the classifier's performance. Later, results are
averaged across all $k$ iterations. It is crucial to remember that
all the steps for building the prediction model should be applied only on the
training data. From this respect, the oversampling (or any kind of data
resampling) technique should not be applied on the whole dataset. Instead, it
is used only on the training dataset generated in each step of the CV
procedure.

A range of hyper-parameters used for the grid search for each of the
classifier and oversampler are presented in the table \ref{tab:grid}.

\begin{table}[H]
	\centering
	\begin{tabular}{lll}
		\toprule
		Classifier       & Hyper-Parameters & Values\\
		\hline
		LR               & maximum iterations   & 10000   \\
		KNN              & number of neighbors  & {3, 5} \\
		DT               & maximum depth        & {3, 6} \\
		GBC              & maximum depth        & {3, 6} \\
    			 		 & number of estimators & {50, 100} \\
		RF               & maximum depth        & {None, 3, 6} \\
				 		 & number of estimators & {50, 100} \\
		\toprule
		Oversampler      &                      & \\
		\hline
		G-SMOTE          & number of neighbors  & {3, 5} \\
				 		 & selection strategy   & combined, minority, majority\\
				 		 & truncation factor    & {-1.0, -0.5, .0, 0.25,
				 		 0.5, 0.75, 1.0} \\
				 		 & deformation factor   & {0, 0.2, 0.4, 0.5,
				 		 0.6, 0.8, 1.0} \\
 		SMOTE            & number of neighbors & {3, 5} \\
		BORDERLINE SMOTE & number of neighbors & {3, 5} \\
		ADASYN           & number of neighbors & {2, 3} \\
		\bottomrule
	\end{tabular}
	\caption{\label{tab:grid}User-defined grid values}
\end{table}


\subsection{Data Set}

For this research LUCAS dataset and Landsat-8 time series from 2015 were used
for training and validation of supervised classifiers. 1694 ground-truth points
from LUCAS 2015 dataset are used as reference data. This dataset is grouped
into 8 classes and represents the main land cover types for the study area. The
area of study is in the continental part of north-western Portugal,
corresponding to the area covered by Landsat image from the track 204 and row
32, with coordinates of center lat: 40.336, long: -8.468.

The dataset's class distribution and imbalance ratio is presented in Table
\ref{tab:dataset_classes}. The data imbalance becomes visible with a minority
class of 4 instances (wetlands) and majority class of 761 instances (woodland).

\begin{table}[H]
	\centering
	\begin{tabular}{llrrrr}
		\toprule
		\textbf{LUCAS Category} & \textbf{Land cover type} & \textbf{Instances}
		& \textbf{Imbalance Ratio} \\
		\hline
		A & Artificial land & 131 & 5.81 \\
		B & Cropland        & 270 & 2.81 \\
		C & Woodland        & 761 & 1.00 \\
		D & Shrubland       & 296 & 2.61 \\
		E & Grassland       & 185 & 4.11 \\
		F & Bareland        & 37  & 20.56 \\
		G & Water           & 10  & 76.10 \\
		H & Wetlands        & 4   & 190.25\\
		\hline
		\textbf{Total} & & \textbf{1694} &  \textbf{190.25} \\
		\bottomrule
	\end{tabular}
	\caption{\label{tab:dataset_classes}LUCAS nomenclature and sample distribution}
\end{table}

The remote sensing dataset presents 8 images from Landsat 8ETM+ multi-spectral
sensor. The images are Level-1 terrain corrected products (L1TP) and are
acquired in 2015 from February to September, 1 image each month. The
acquisition mode is Descending. Only bands 2, 3, 4, 5, 6 and 7 are used from
each image. Thus, each reference point from LUCAS dataset has 48 features
(Table \ref{tab:datasets}). Those features have the value of digital numbers of
the pixels from each spectral bands of the image time series representing LUCAS
point location.

\begin{table}[H]
	\centering
	\csvautobooktabular{./resources/dataset_descrip.csv}
	\caption{\label{tab:datasets}Description of the dataset}
\end{table}

\section{Results and Discussion}

This section presents the final results of classifiers performance with
different oversamplers over LUCAS dataset. A simple ranking of oversamplers
performance is carried out for comparative analysis.

The cross-validation scores for each combination of classifiers, evaluation
metrics and oversamplers are presented in Table \ref{tab:mean_cross_val}.
From this table we can observe that Geometric-SMOTE over-performs nearly all
other oversampling methods for both G-mean and F-score metrics on all
classifiers. The absolute best results are achieved when Geometric-SMOTE is
combined with Fandom Forest.

\begin{table}[H]
	\centering
	\csvautobooktabular{./resources/mean_scores.csv}
	\caption{\label{tab:mean_cross_val}Results for cross validation scores
		of oversamplers (NO OS: No Oversampling, RAND OS: Random Oversampling, B-SMOTE: Borderline SMOTE, G-SMOTE: Geometric SMOTE)}
\end{table}

Amongst the classifiers Random Forest shows best performance for all three
measures. It recorded constantly improved results for both G-mean and F-score
metrics while combined with oversamplers. Second best classification results
are obtained with Gradient Boosting Classifier, and the worst performance is
with single Decision Tree Classifier.

Somewhat contrary to expectations, for some of the data balancing methods
slightly decreased the accuracy of imbalance-specific metrics while compared
to the baseline methods. For instance, the KNN classifier recorded higher G-
mean value (0.50) with unchanged data compared to the results of Random
Oversampling (0.48), ADASYN (0.48) and SMOTE (0.49). Similarly, Logistic
Regression classifier recorded higher F-score value (0.30) with No Oversampling
compared to the ones with Random Oversampling (0.29), SMOTE (0.29) and ADASYN
(0.28).

A ranking score was assigned to each oversampling method with the best and
worst performing methods receiving scores equal to 1 and 6, respectively. This
comparison is presented in Table \ref{tab:ranking}. From this table we can
observe that G-SMOTE systematically outperforms any other oversampling
algorithm according to the G-mean measure. Using F-score it performs best on
all classifiers except the Decision Tree classifier, where Borderline-SMOTE
outperforms the remaining oversamplers. Regardless, its discrepancy with
Geometric-SMOTE is negligible.

\begin{table}[H]
	\centering
	\csvautobooktabular{./resources/mean_ranking.csv}
	\caption{\label{tab:ranking} Results for ranking of oversamplers (NO OS: No Oversampling, RAND OS: Random Oversampling, B-SMOTE: Borderline SMOTE, G-SMOTE: Geometric SMOTE)}
\end{table}

The mean of ranking scores for each of oversampling method across the
classification algorithms is presented in the Table \ref{tab:mean_rankings}.
From this table it is easier to interpret that Geometric-SMOTE has been the
best method almost all the time for G-mean and F-score measures. The same way
Borderline-SMOTE has been the second best method almost all the times. In
contrast to those two methods, in average ADASYN could not improve the
classification results for any of measures compared to SMOTE method.

Accuracy scores' results depict the bias discussed in section \ref{evaluation
methods}. This metric, by attributing proportional importance to the
predictive power of each class according to the dataset's class distribution,
hinders the importance of minority classes. Therefore, this metric can be
misleading for imbalanced data classification problems. Accordingly, models
with no oversampling present the highest accuracy score. In a multi-class
classification problem with an imbalanced dataset, where the prediction of all
the classes are of equal importance, overall accuracy should not be implemented
to evaluate the model performance, as is commonly utilised in remote sensing.
However, even for accuracy metric, Geometric-SMOTE shows the best performance
amongst the data modification algorithms and exceeds the rest of the
oversampling methods.


\begin{table}[H]
	\centering
	\csvautobooktabular{./resources/model_mean_ranking.csv}
	\caption{\label{tab:mean_rankings} Results for mean ranking of oversamplers}
\end{table}


Table \ref{tab:smote_gsmote} presents the percentage difference  between
Geometric-SMOTE and SMOTE. For this comparison SMOTE algorithm was selected due
to its popularity and wide use within the remote sensing society (see section
2). The table shows that Geometric-SMOTE outperforms SMOTE regardless of the classifier or metric used.

\begin{table}[H]
	\centering
	\csvautobooktabular{./resources/mean_perc_diff_scores.csv}
	\caption{\label{tab:smote_gsmote}Results for percentage difference between
		G-SMOTE and SMOTE}
\end{table}

\section{Conclusions}

In this paper we presented the performance of Geometric-SMOTE oversampling
algorithm as a novel approach to deal with class imbalance problem in naturally
imbalanced remote sensing datasets  with an aim to improve the LULC
classification. Geometric-SMOTE's performance was evaluated and compared with
other oversampling methods, specifically Random Oversampling, SMOTE and its
variations Borderline-SMOTE and ADASYN, through the use of a number of
classifiers: Logistic Regression, K-Nearest Neighbours, Decision Trees,
Gradient Boosting Classifier and Random Forests.

For this purpose, an imbalanced LUCAS dataset was used. For model validation
and comparison Overall Accuracy was used, as well as G-mean and F-score as they
represent imbalance-specific metrics. The models were trained using the 3-fold
cross-validation method. The experiment results show the importance of
oversampling methods for multiclass imbalanced data classification as in most
of the cases balancing data with oversampling resulted for higher G-mean and F-
score values.

Geometric-SMOTE shows the best results and dominated the ranking across the
oversamplers. However, table \ref{tab:ranking} reports Decision Trees to
achieve a higher F-Score when using the Borderline SMOTE algorithm. Although
this algorithm systematically performs better than any other oversampling
method using G-mean and F-score metrics, it may be useful to also consider
Borderline SMOTE. Geometric-SMOTE proves to be a useful tool for remote sensing
researchers and can effectively replace the two widely accepted oversamplers
Random Oversampling and SMOTE methods.


\bibliography{references}
\bibliographystyle{apalike}

\end{document}
